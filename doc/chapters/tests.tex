%
% tests.tex
%
% Copyright (C) 2021 by SpaceLab.
%
% EPS 2.0 Documentation
%
% This work is licensed under the Creative Commons Attribution-ShareAlike 4.0
% International License. To view a copy of this license,
% visit http://creativecommons.org/licenses/by-sa/4.0/.
%

%
% \brief Test procedures chapter.
%
% \author Yan Castro de Azeredo <yan.ufsceel@gmail.com>
%
% \institution Universidade Federal de Santa Catarina (UFSC)
%
% \version 0.2.0
%
% \date 2021/04/24
%

\chapter{Test Procedures} \label{ch:test-procedures}

This chapter follows a standard workflow created by SpaceLab for testing its CubeSat modules; these procedures are first referenced in FloripaSat-2 documentation chapter 7\cite{floripasat2-doc}. The \autoref{tab:test-procedures-table} resumes the workflow and each one of the test types, subtests, and code identification. Some subtests can be considered to be not applied to the platform; the order in which they are done does not need to follow the numeration, and others have generic titles that don't require too much explanation to be accomplished. The particularities of the EPS 2.0 for each test type are described in the following sections, and the results of those tests in the v0.1 and v0.2 versions of the EPS 2.0 are presented in the \autoref{anx:test-report-v01} and \autoref{anx:test-report-v02}.

\begin{table}[!h]
    \centering
    \begin{tabular}{l|p{105mm}|p{5mm}}
        \toprule[1.5pt]
        Test type     & Subtests & ID \\
        \midrule
        A. Visual Inspection     & 1. Packaging quality assessment \newline 2. Board manufacturing and assembly quality \newline 3. 3D model comparison \newline 4. Layers marker \newline 5. Labels (schematics comparison) \newline 6. High resolution photos for documentation & TA1 \newline TA2 \newline TA3 \newline TA4 \newline TA5 \newline TA6 \\
        \midrule
        B. Mechanical Inspection     & 1. Board dimensions and mounting holes positioning \newline 2. Board weight measurement & TB1 \newline TB2 \\
        \midrule
        C. Integration Inspection    & 1. Check connectors pinout against the documentation \newline 2. Check connectors positioning & TC1 \newline TC2 \\
        \midrule
        D. Electrical Inspection     & 1. Solder shorts \newline 2. Missing components \newline 3. Lifted pins \newline 4. Poor soldering \newline 5. Swapped components \newline 6. Components partnumber & TD1 \newline TD2 \newline TD3 \newline TD4 \newline TD5 \newline TD6 \\
        \midrule
        E. Electrical Testing       & 1. Continuity test \newline 2. Power up procedures \newline 3. Average input power consumption measurement \newline 4. Average output power source measurement \newline 5. Power tracks temperature \newline 6. Simple signal integrity & TE1 \newline TE2 \newline TE3 \newline TE4 \newline TE5 \newline TE6 \\
        \midrule
        F. Functional Testing     & 1. Simple test code run \newline 2. System code run \newline 3. System hardware self-test flags check \newline 4. Monitor LEDs behavior \newline 5. Monitor the debug serial port logs & TF1 \newline TF2 \newline TF3 \newline TF4 \newline TF5 \\
         \midrule
        G. Module Testing     & 1. Review operation behavior \newline 2. Review features and requirements fulfillment \newline 3. Review communication buses configuration and protocol \newline 4. Review data packages, power buses, and control signals \newline 5. Review edge cases and evaluate damage \newline 6. Run remote automated code tests \newline 7. Run system test codes in the board \newline 8. Run latest stable code version and review behavior & TG1 \newline TG2 \newline TG3 \newline TG4 \newline TG5 \newline TG6 \newline TG7 \newline TG8  \\
        \bottomrule[1.5pt]
    \end{tabular}
    \caption{Test workflow table.}
    \label{tab:test-procedures-table}
\end{table}

\section{Visual Inspection}

The first step when receiving a manufactured (and in some cases assembled) PCB is to inspect it visually to see if it is according to its expected appearance. The EPS 2.0 has more than 300 components, so it is advised for this test not to take a long time to accomplish; the major noted problems should be spotted and reported. The \autoref{tab:visual-inspection} resumes the visual inspection steps. 

\begin{table}[!htb]
\centering
\caption{Visual Inspection test steps.}
\label{tab:visual-inspection}
\begin{tabular}{m{3cm} m{12cm} m{3cm}}
\toprule
Test type & Visual Inspection \\
\midrule
\midrule
subtests code & Description \\ 
\midrule
TA1 & Verify the received package, review the packaging protection used, and if it maintained the physical and/or aesthetic integrity of the board. \\
\midrule
TA2 & See if the overall quality specified on the manufacturing process is according with its IPC class, the engineering model should be Class 2 while the flight model Class 3. \\
\midrule
TA3 & Inspect the top and bottom side of the PCB and verify if all the components are well soldered and if their polarity and labels are correct according to schematics and 3D model. Figure \ref{fig:pcb-top} and \ref{fig:pcb-bottom} for reference. If the board is not yet assembled detail, the inconsistencies found during the assembly process. \\
\midrule
TA4 & Take a high-resolution centered photo of both sides of the board for documentation; avoid reflections if possible. \\
\midrule
\midrule
Success Criteria & The board needs to appear functional regarding its visual electrical specs. \\
\bottomrule
\end{tabular}
\end{table}

\section {Mechanical Inspection}

These tests verify if the board has the expected mechanical specs before integration. The \autoref{tab:mechanical-inspection} resumes the mechanical inspection steps. 

\begin{table}[!htb]
\centering
\caption{Mechanical Inspection test steps.}
\label{tab:mechanical-inspection}
\begin{tabular}{m{3cm} m{12cm} m{3cm}}
\toprule
Test type & Mechanical Inspection \\
\midrule
\midrule
subtests code & Description \\ 
\midrule
TB1 & Verify board dimensions outline and mounting holes size and positioning with a measurement tool or through the board's draftsman sheet 1:1 scale print. \\
\midrule
TB2 & Measure board weight with an electronic balance with at least 0.1 grams of precision. \\
\midrule
\midrule
Success Criteria & The fabricated board needs to follow the mechanical specification of the draftsman document. \\
\bottomrule
\end{tabular}
\end{table}

\section{Integration Inspection}

These tests verify the integration before the module's full assembly on the CubeSat. The \autoref{tab:integration-inspection} resumes the integration inspection steps. 

\begin{table}[!htb]
\centering
\caption{Integration Inspection test steps.}
\label{tab:integration-inspection}
\begin{tabular}{m{3cm} m{12cm} m{3cm}}
\toprule
Test type & Integration Inspection \\
\midrule
\midrule
subtests code & Description \\ 
\midrule
TC1 & Check external connectors pinout labels against the documentation present on \autoref{external-connectors}. \\
\midrule
TC2 & Check connectors positioning and possible integration issues from connected cables with the mounted BAT4C board and solar panels. \\
\midrule
\midrule
Success Criteria & The labels and placement of the external connectors must be according to the documentation and not present any possible integration issues. \\
\bottomrule
\end{tabular}
\end{table}

\section {Electrical Inspection}

These tests verify if the components used in the mounted module are correct, if they are well-soldered and if there is any other error in the layout. \autoref{tab:electrical-inspection} resumes the electrical inspection steps.

\begin{table}[!htb]
\centering
\caption{Electrical Inspection test steps.}
\label{tab:electrical-inspection}
\begin{tabular}{m{3cm} m{12cm} m{3cm}}
\toprule
Test type & Electrical Inspection \\
\midrule
\midrule
Subtests code & Description \\ 
\midrule
TD1 & Check for any visible short circuit in the module buses. It is not necessary to use any equipment; it is a ``visual inspection" only. \\
\midrule
TD2 & Check if all the components were soldered. If there are missing components, it is necessary to evaluate if it was intended or not. \\
\midrule
TD3 & Check if the pins of the ICs and other components are damaged. \\
\midrule
TD4 & Check if there is any poor soldering. \\
\midrule
TD5 & Check if the components are swapped or connected ``backward". \\
\midrule
TD6 & Check the part number of the components. \\
\midrule
\midrule
Success Criteria & All the intended components are connected correctly to the board, without any visible errors. \\
\bottomrule
\end{tabular}
\end{table}


\section {Electrical Testing}

Now it is necessary to use equipment such multimeter, oscilloscope, power supply, or anything that can be used to evaluate the interface between components and connectors. \autoref{tab:electrical-testing} resumes the electrical testing steps.

\begin{table}[!htb]
\centering
\caption{Electrical Testing steps.}
\label{tab:electrical-testing}
\begin{tabular}{m{3cm} m{12cm} m{3cm}}
\toprule
Test type & Electrical Testing \\
\midrule
\midrule
Subtests code & Description \\ 
\midrule
TE1 & Using a multimeter, check for any short circuit in the buses of the module, primarily the power buses. \\
\midrule
TE2 & If the module does not present short circuits between GND and VCC signals, power up the module with a power supply to see if it turns on. \\
\midrule
TE3 & Check the power consumption of the module. \\
\midrule
TE4 & Check if the converters of the module are working as intended. \\
\midrule
TE5 & Check if the module is not overheating. \\
\midrule
TE6 & Check if the signals are being transmitted correctly through the buses. \\
\midrule
\midrule
Success Criteria & The modules do not present any short circuits and can be turned on without any critical problems. \\
\bottomrule
\end{tabular}
\end{table}

\section {Functional Testing}

It is necessary to evaluate if the module can run any code and if it is ready to be used during FSW development and debugging. \autoref{tab:functional-testing} resumes the functional testing steps.

\begin{table}[!htb]
\centering
\caption{Functional Testing steps.}
\label{tab:functional-testing}
\begin{tabular}{m{3cm} m{12cm} m{3cm}}
\toprule
Test type & Functional Testing \\
\midrule
\midrule
Subtests code & Description \\ 
\midrule
TF1 & Check if the module can run a code. Preferably something straightforward, such as a ``blinking LED". \\
\midrule
TF2 & Check if the module can run its firmware, even if it is incomplete. \\
\midrule
TF3 & Check if the module can make some self-tests. The results of these self-tests are going to be presented in its log. \\
\midrule
TF4 & Check if the LEDs are working as intended. \\
\midrule
TF5 & Check the logs for a longer period. \\
\midrule
\midrule
Success Criteria & The module has minimal expected functionalities. \\
\bottomrule
\end{tabular}
\end{table}
 
\section {Module Testing}

The module will be tested to evaluate if it can communicate with other modules, if its sensors are measuring all the variables correctly (e.g., voltage, current, temperature), and if it is operating as expected. \autoref{tab:module-testing} resumes the module testing steps.

\begin{table}[!htb]
\centering
\caption{Module Testing steps.}
\label{tab:module-testing}
\begin{tabular}{m{3cm} m{12cm} m{3cm}}
\toprule
Test type & Functional Testing \\
\midrule
\midrule
Subtests code & Description \\ 
\midrule
TG1 & Run and review the module's firmware and evaluate it using any equipment that SpaceLab offers. \\
\midrule
TG2 & Check and review the drivers, devices, and tasks to see if the features are correctly implemented. \\
\midrule
TG3 & Check and review the configuration of the communication protocols used. \\
\midrule
TG4 & Check and review all the buses to see if there is any anomaly or unexpected behavior. \\
\midrule
TG5 & Check and review the module's operation when submitting an edge situation. \\
\midrule
TG6 & Run the unit tests. \\
\midrule
TG7 & Run system test codes in the board. \\
\midrule
TG8 & Run the latest stable code version and review behavior. \\
\midrule
\midrule
Success Criteria & The is operating correctly. \\
\bottomrule
\end{tabular}
\end{table}
