%
% instructions.tex
%
% Copyright (C) 2021 by SpaceLab.
%
% EPS 2.0 Documentation
%
% This work is licensed under the Creative Commons Attribution-ShareAlike 4.0
% International License. To view a copy of this license,
% visit http://creativecommons.org/licenses/by-sa/4.0/.
%

%
% \brief Instructions chapter.
%
% \author Gabriel Mariano Marcelino <gabriel.mm8@gmail.com>
%
% \institution Universidade Federal de Santa Catarina (UFSC)
%
% \version 0.2.0
%
% \date 2020/11/05
%

\chapter{Usage Instructions} \label{ch:instructions}

\section{Powering the Board}

The EPS 2.0 is the energy provider module within a satellite; its nominal operation is alongside a battery pack from the BAT4C module and solar panels. 

During development and testing, the board can also be partially powered using the JTAG interface (only the MCU and peripherals connected to the 3V3 EPS regulator) or fully powered directly through the battery connector with an equivalent power supply.

The module's PC104 power pins can be accessed externally, but it is advised to be used only for test probes and not powering directly from them. 

In the following subsections powering from the JTAG interface and batteries connector are explained in detail. 
It is advised to have the kill switches on the "NO" position and/or the RBF switch active before connecting the probes or cables to power the module.

% add real image of the IIP been used with EPS2
 
\subsection{Powering through JTAG Interface}

The JTAG interface is used for programming and debugging the module using a Flash Emulation Tool (refer to \autoref{jtag-picoblade} on the hardware chapter). 
This is done using a MSP-FET with the Spy-Bi-Wire protocol.
The tool should provide $3.3 V$ and a maximum of $100\ mA$; due to this current limitation, only the EPS MCU can be powered for minimal testing and debugging purposes; for this, the CN1 header pins must be shorted.
Remember that the CN1 connector should only be used if the EPS board is not powered by other means like batteries or external power supply.

For interfacing the 14 pin cable of the MSP-FET to the EPS, an adapted cable or an external interface is required. 
The IIP\cite{iip} Nº1 and Nº2 boards have a 14 pin header that translates in a picoblade connector for the required connector counterpart on the EPS module; any of the pin header slots from 1 to 4 can be used.
When the MSP-FET is correctly connected and the necessary cable connections are made, the kill switches can be put on the "NC" position, and/or the RBF switch can be released.
The 3V3 MCU (refer to \autoref{status-leds} on the system overview chapter) indicates that the 3.3V power is being sourced; the system on led can be checked to see any easily detectable misbehavior right after the programming of the board. 
%On \autoref{eps-iip-connection} shows the connection between EPS and IIP.


% add real image of the IIP been used with EPS2

\subsection{Powering through Power Supply}

Two voltage sources are needed to power the EPS module through an external power supply.
The external power sources can be set up in two ways, either two \(3.7 V\) sources in series or a \(3.7 V\) source for reference and a \(7.4 V\) source for power.
In both cases, the voltage sources should be connected to one of the battery board connectors (P1 or P5).

To use two \(3.7 V\) sources, the first source should have it's negative and positive terminals connected to Vbat- and Vbat\_common pins of the batteries connector respectively, the second source should have it's negative and positive terminals connected to Vbat\_common and Vbat+ pins of the batteries connector respectively (refer to \autoref{fig:battery-connector} and \autoref{tab:battery-connector} on the hardware chapter).

To use \(3.7 V\) and \(7.4 V\) sources, the \(3.7 V\) source should have its negative and positive terminals connected to Vbat- and Vbat\_common pins of the batteries connector, respectively, the \(7.4 V\) source should have it is negative and positive terminals connected to Vbat- and Vbat+ pins of the batteries connector respectively (refer to \autoref{fig:battery-connector} and \autoref{tab:battery-connector} on the hardware chapter).

Suppose the external power supply has configurable current limitation capability. In that case, it should be adjusted accordingly to the expected consumption of the load connected to the EPS module (other modules being powered by the EPS, batteries heaters being active, etc.); if powering the EPS in isolation (no load connected) a limit of \(80 mA\) may be applied.

After correctly connecting the voltage sources, the kill switches can be put on the "NC" position, and/or the RBF switch can be released.

% add appendix instructions for equivalent power supply powering

\subsection{Powering through Batteries}

To power the EPS module from the batteries, the BAT4C module must be connected through the P5 connector (refer to \autoref{fig:battery-connector} on the hardware chapter).
The batteries can be charged if needed using the P3 PicoBlade connector (refer to \autoref{external-charge-picoblade} on the hardware chapter), it will also require an external interface or an adapted cable to be used for interfacing the charger device to the PicoBlade.
The batteries also can be charged directly from the BAT4C module, for this refer to its documentation on the usage instruction chapter for more details \cite{bat4c}.


\section{Log Messages}

The EPS 2.0 has a UART interface for debugging, described in \autoref{tab:usci-config}. It follows a log system structure to improve the information provided in each message.
The messages can be acquired by connecting a USB cable to the IIP Nº3 board with an integrated FTDI FT4232H IC \cite{iip}.
