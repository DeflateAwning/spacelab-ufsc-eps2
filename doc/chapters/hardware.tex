%
% hardware.tex
%
% Copyright (C) 2020 by SpaceLab.
%
% EPS 2.0 Documentation
%
% This work is licensed under the Creative Commons Attribution-ShareAlike 4.0
% International License. To view a copy of this license,
% visit http://creativecommons.org/licenses/by-sa/4.0/.
%

%
% \brief Hardware project chapter.
%
% \author Gabriel Mariano Marcelino <gabriel.mm8@gmail.com>
%
% \institution Universidade Federal de Santa Catarina (UFSC)
%
% \version 0.1.0
%
% \date 2020/11/05
%

\chapter{Hardware} \label{ch:hardware}

.

\section{MPPT Boost Converters}

There are three boost converters in the system, one for each couple of solar panels in parallel connection. Each one is a discrete boost with a HC9-220-R inductor, a SI4010dy mosfet as the switch and a B340LA-13-F diode. There are six GRM32ER1E226KE15L capacitors and two GRM216R71H103KA01D capacitors connected in parallel in the boost output. The output filter is the same for all the converters as their outputs are tied together. The control PWM\nomenclature{\textbf{PWM}}{\textit{Pulse Width Modulation.}} signals are generated by the MCU at a frequency of nearly 500 kHz. Finally, the EPS PCB is provided with a LMC555 chip, which is able to generate a fixed PWM for the MPPT\nomenclature{\textbf{MPPT}}{\textit{Maximum Power Point Tracking.}} circuit in case of EPS MCU\nomenclature{\textbf{MCU}}{\textit{Microcontroller Unit.}} failures.

\section{Measurement Circuits}

The measurement circuits are used to generate a voltage proportional to the variable being measured, in a range accepted by the MCU internal ADC\nomenclature{\textbf{ADC}}{\textit{Analog to Digital Converter.}}.

\subsection{Solar Panels Current}

The main component of the solar panels currents measurement circuit is the MAX9934TAUA+ current sense amplifier. It generates an output current proportional to the differential input voltage. The gain is 25 $\mu$A/mV. To make the measurements possible, the current goes through 50 m$\Omega$, 0.5 \% resistors, connected to the inputs of the amplifier, and the outputs are connected to 3.3 k$\Omega$ resistors. The output voltage of the circuit is given by:

\begin{equation}
V_{out} = I_{sense} \cdot R_{sense} \cdot G \cdot R_{out}
\end{equation}

\subsection{Beacon Current}

This measurement takes place at the output of the EPS-Beacon regulator. It also uses a MAX9934TAUA+ current sense amplifier, but with a shunt resistor of 75 m$\Omega$, 0.5 \% and the output connected to a 4.02 k$\Omega$ resistor.

\subsection{Solar Panels Voltage}

The solar panels voltage measurement circuit is composed by a voltage divider and an op-amp in a buffer configuration. The voltage divider is composed of a 93.1 k$\Omega$ resistor and an 100 k$\Omega$ resistor. The op-amp is a TLV341AIDBVR chip. The output voltage is given by:

\begin{equation}
V_{out} = V_{sp} \cdot \frac{R_{2}}{R_{1} + R_{2}}
\end{equation}

\subsection{Boost Converters Output Voltage}

The boost converters output voltage measurement circuit is very similar to the solar panels voltages measurement circuit, with the exception that the voltage divider is composed by a 300 k$\Omega$ resistor and an 100 k$\Omega$ resistor.

\subsection{Main Power Bus Voltage}

The main power bus voltage measurement circuit is identical to the boost converters output voltage measurement circuit.

\section{Heaters Control Circuit}

The batteries operate over a specified temperature range and need active heating to work properly in space. The heaters control circuit is composed of the heaters themselves, RTDs\nomenclature{\textbf{RTD}}{\textit{Resistive Temperature Detector.}}, an external ADC and the drivers.

\subsection{ADC}

The ADS1248 chip generates a precise reference current to the RTDs, and samples the voltage proportional to the temperature established over the sensors. This voltage is converted to digital data and sent to the MCU via SPI protocol.

\subsection{Heaters Drivers}

The drivers are chopper converters controlled by the MCU, with a PWM frequency of 50 kHz. The switches of the chopper converters are Si4010DY mosfets.

\section{Kill-Switches}

These switches are used to separate the solar panels and the batteries from the load during pre-flight and launch. Each one is composed of two SI4403-CDY-T1-GE3 P-channel mosfets in parallel, as a redundancy. When either the RBF\nomenclature{\textbf{RBF}}{\textit{Remove Before Flight.}} is in place or the kill-switches are pressed, the mosfets disconnect the loads from the sources.

\section{Voltage Regulators}

To supply itself and the other modules, the EPS has 6 integrated DC-DC regulators. To supply the Beacon MCU and itself a TPS5420QDRQ1 regulator is used, with and output voltage of 3.3 V and 2 A current capability. This regulator is always on.

To power the payload, a TPS5430QDDARQ1 regulator is used. It has an output voltage of 5 V and 3 A current capability. The EPS can enable/disable this regulator.

OBDH and the main radio are powered by TPS5410QDRQ1 regulator, with an output voltage of 3.3 V and 1 A current capability. The EPS can enable/disable this regulator.

The antenna deployment system has a dedicated regulator (TPS5420QDRQ1), with 3.3 V output voltage and 2 A current capability. This regulator is always on.

Finally, each PA is powered by its own TPS54540QDDARQ1 regulator, with an output voltage of 5 V and 5 A current capabiity. The Beacon MCU controls its PA regulator enable/disable function and the OBDH MCU controls the main radio PA regulator enable/disable function.
